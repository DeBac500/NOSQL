\documentclass[a4paper,12pt]{scrreprt}
\usepackage[T1]{fontenc}
\usepackage[utf8]{inputenc}
\usepackage[ngerman]{babel}
\usepackage[table]{xcolor}% http://ctan.org/pkg/xcolor
\usepackage{tabu}
\usepackage{graphicx}
\usepackage{lmodern}
\usepackage{hyperref}
\usepackage{mathrsfs}

\begin{document}


%\titlehead{Kopf} %Optionale Kopfzeile
\author{Alexander Rieppel \and Dominik Backhausen} %Zwei Autoren
\title{ NOSQL } %Titel/Thema
\subject{VSDB} %Fach
%\subtitle{ } %Genaueres Thema, Optional
\date{\today} %Datum
\publishers{5AHITT} %Klasse

\maketitle
\tableofcontents


\chapter{Aufgabenstellung}
Entwerfen Sie ein System zur Verwaltung von alten Prüfungsfragen und Aufgabenstellungen. Beachten Sie dabei eine einfache und modulare Art der Datenspeicherung. Es soll die Möglichkeit bestehen, Fragen zu unterschiedlichen Hauptkategorien zuzuordnen (Unterrichtsfächer bzw. Kompetenzbereich, z.B. INSY, SYT bzw. "Dezentrale Systeme" etc.). Weiters soll es aber auch möglich sein den Fragen und Aufgaben weitere Informationen beizufügen (Tags, z.B. JEE, Datawarehouse, XML, Hibernate etc.). Fragen werden Kandidaten bzw. Klassen zur Vorbereitung an einem bestimmten Termin zugewiesen. Die Liste aller Fragen und Aufgaben soll sortierbar sein und als solche auch ausgedruckt werden können. Da die endgültige Definition der Archivierung nicht abgeschlossen ist, muss eine einfache Erweiterungsmöglichkeit gegeben sein.\\\\Die Anzeige und Verwaltung der Fragen soll nur authentifizierten Benutzern gestattet werden.\\\\Achten Sie beim Design und der Implementierung des Systems auf Modularität und Erweiterbarkeit. Verwenden Sie für die graphische Visualisierung ebenfalls gängige Frameworks, wie zum Beispiel Bootstrap . Verwenden Sie für die Implementierung ein passendes Web-Framework (Python, Java, PHP, ...) und ein der Aufgabenstellung entsprechendes NoSQL-Datenbankmanagementsystem (MongoDB, ...).\\\\Gruppenarbeit von 2 bis 4 Mitgliedern, wobei die Anforderungen (graphische Oberfläche, funktionale Anforderungen) der Anzahl der Gruppenmitgliedern angepasst wird. Erweiterungen werden in Klammern entsprechend deklariert (z.B. 4er Gruppe muss es zusätzlich implementieren, die anderen müssen es nicht realisieren aber die technische Möglichkeit im Protokoll erörtern), sonst ist es ein Must-Have-Requirement.
\section{Anforderungen}
\begin{itemize}
\item Anzeige / Editierbarkeit [multiplizität]
\subitem Aufgabenart (Maturafrage, praktische Aufgabenstellung, schriftliche Mitarbeitsüberprüfung, Jahresprüfung, ...) [1..*]
\subitem Kategorie (SYT, INSY, Dezentrale Systeme, ...) [1..*]
\subitem Tags (JEE, Hibernate, XML, ...) [0..*]
\subitem Author [1..*]
\subitem erstellt / zuletzt geändert [1]
\subitem zuletzt verwendet [0..*]
\subitem zugeteilt [0..*]
\subitem Angabentext [1]
\subitem Hinzufügen von zusätzlichen Keys und deren Values (4er)
\subitem Zusätzliche Attachments (Bilder, Videos, etc.) (3er)
\item Suche
\subitem Aufgabenart, Kategorie und Tags
\subitem Datum
\subitem Volltextsuche über alle Values (4er)
\item Administration
\subitem Benutzer verwalten
\subitem Logging von Benutzeraktionen (4er)
\subitem Snapshots und Wiederherstellung der Datenbasis (3er)
\end{itemize}
\chapter{Designüberlegungen}
\section{Zu implementierende Anforderungen}
Als Datenbank wird MongoDB verwendet und für die Administrationsoberfläche wird ein Servlet erstellt, welches in Java EE implementiert wird. Für die Speicherung von Aufgabenart, Kategorie, Tags, Autor und Angabentext, werden simple Strings in der Datenbank abgelegt. Für erstellt/zuletzt geändert sowie zuletzt verwendet wird ein Datum im entsprechenden Format abgespeichert. Die Suche %TODO
Die Benutzerverwaltung läuft über eine eigene Administrationsseite ab die über den Browser aufgerufen werden kann. 

\section{3er- und 4er-Gruppen Anforderungen}
Für das Hinzufügen von zusätzlichen Keys und deren Values wird der Datensatz  gesucht und ein update auf diesen durchgeführt. Für zusätzliche Attachments von z.B. Bildern und Videos, wird einfach ein Button erstellt hinter den das entsprechende File per link angehängt wird. Dies könnte unter Umständen auch ein ftp Link sein. Die Volltextsuche %TODO
Das Logging der Benutzeraktionen könnte z.B. über ein eigenes Framework realisiert werden (Log4j). Snapshots und Wiederherstellung der Datenbasis %TODO


\chapter{Arbeitsaufteilung}
	\tabulinesep = 4pt
	\begin{tabu}  {|[2pt]X[2.5,c] |[1pt] X[4,c] |[1pt]X[1.3,c]|[1pt]X[c]|[2pt]}
		\tabucline[2pt]{-}
		Name & Arbeitssegment & Time Estimated & Time Spent\\\tabucline[2pt]{-}
		Alexander Rieppel & Datenbank installieren & 1h & 0.5h\\\tabucline[1pt]{-}
		Alexander Rieppel & Datenbank konfigurieren & 1h & 1h\\\tabucline[1pt]{-}
		Alexander Rieppel & JavaEE Servlet schreiben & 2h & 1h\\\tabucline[1pt]{-}
		Dominik Backhausen & JavaEE Servlet schreiben & 2h & 3h\\\tabucline[1pt]{-}
		Dominik Backhausen & Datenbank Schnittstelle schreiben& 2h & 1.5h\\\tabucline[2pt]{-}
		Gesamt && 8h & 7h\\\tabucline[2pt]{-}
	\end{tabu}	
	
\chapter{Arbeitsdurchführung}
\section{Resultate}
\begin{itemize}
\item MongoDB wurde erfolgreich installiert und konfiguriert
\item Das Servlet wurde funktionalistisch so wie in der Designüberlegung angegeben implementiert
\item Die Speicherung der Strings funktioniert problemlos
\item Die Daten für erstellt, zuletzt verwendet etc. werden einwandfrei angezeigt und gespeichert
\item Die Suche funktioniert so wie in der Designüberlegung angegeben
\item Die Benutzerverwaltung läuft über eine Adminseite
\end{itemize}
\section{Niederlagen}
\begin{itemize}
\item Bei der Darstellung des Servlets im Browser wurde auf ein externes Framework verzichtet und reines html verwendet

\end{itemize}
\chapter{Testbericht}
%TODO
\chapter{Quellen}
\href{http://docs.mongodb.org/manual/reference/}{[1]http://docs.mongodb.org/manual/reference/}\\
\href{https://www.mongodb.org/}{[2]https://www.mongodb.org/}\\
\href{http://api.mongodb.org/java/}{[3]http://api.mongodb.org/java/}
\end{document}